\section{Objetivos del proyecto}
% Los objetivos deben formularse de manera que logren transmitir lo que intenta realizar el investigador y lo que espera obtener como resultado. 

% Los objetivos deben iniciar con un verbo en infinitivo (construir, diseñar, seleccionar, analizar, modelar simular, etcétera.) 

% En resumen el objetivo general responde \textbf{¿qué se espera lograr con el proyecto de grado?}


\subsection{Objetivo General}
Diseñar e implementar una infraestructura MLOps para el despliegue o entrega continua de un modelo de Machine Learning para el reconocimiento y control de las enfermedades Stenoma catenifer y heilipus lauri en el cultivo de aguacate Hass.




\subsection{Objetivos específicos}
\begin{itemize}
  \item Desarrollar una infraestructura MLOps que permita la integración, automatización y monitorización del modelo de machine learning para el reconocimiento y control de las enfermedades Stenoma catenifer y heilipus lauri en el cultivo de aguacate Hass.
  \item Validar el uso de MLOps mediante despliegue en un ambiente controlado, con la capacidad de monitorear y mejorar continuamente el rendimiento del modelo.
  \item Desarrollar y entrenar un modelo de Machine Learning utilizando técnicas apropiadas de preprocesamiento y selección de características, así como algoritmos de aprendizaje supervisado o no supervisado, para lograr una detección de las enfermedades Stenoma catenifer y heilipus lauri en el cultivo de aguacate Hass.
  \item Implementar técnicas de procesamiento de imágenes para extraer características relevantes y mejorar la capacidad del modelo de Machine Learning en el reconocimiento y detección de las enfermedades Stenoma catenifer y heilipus lauri en el cultivo de aguacate Hass a partir de imágenes capturadas en campo.
\end{itemize}



%\subsection{Resultados esperados}
%Los resultados esperados se redactan teniendo en cuenta los objetivos de investigación, el problema que se quiere investigar, y las posibilidades reales de producir los mismos reconociendo las condiciones en que puede operarse o ejecutarse el proyecto de investigación.