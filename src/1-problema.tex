\section{Definición del problema}
% Un problema es todo aquello cuya solución se desconoce; ese desconocimiento puede ser para un grupo de personas o para la humanidad. Para la formulación correcta de un problema se debe tener en cuenta los siguientes aspectos:
 
% \textcolor{red}{Como decía \citet{Glorot2011} el trabajo deberia hacerse de la siguiente manera:} 

\subsection{Planteamiento del problema}
Los aportes económicos, laborales, alimenticios y sociales de los procesos agrícolas son fundamentales para el desarrollo de un país, situación que pone de manifiesto la mirada de los entes gubernamentales y no gubernamentales internacionales y nacionales en el renglón de la economía agrícola y de la comunidad que se encuentra articulada en este ámbito económico.\\ 

La producción agrícola del aguacate Hass para el caso de México en el año 2019 correspondió a un 2.4 millones de toneladas, aportando el 45\% en las exportaciones de este país y aumentando la cantidad de exportaciones en un 22\% en el año 2020 \citet{cruz2022competitividad}. Estos aportes a nivel nacional se reflejan en PIB, contribuyendo al avance socioeconómico de las regiones agrícolas.\\

El aguacate Hass corresponde cerca del 82\% de todos los aguacates el más consumido a nivel mundial. De acuerdo con los datos de la Organización de las Naciones Unidas para la Alimentación y la Agricultura \citet{faostat2021hacia} el primer país productor de aguacate Hass es México con unas 2.393.849 toneladas al año, seguido de Colombia con unas 876.754 toneladas al año y de República Dominicana con 676.373 toneladas al año.\\

Los avances en la agroindustria en Colombia contribuyen al ámbito económico y laboral, siendo en la actualidad la producción agrícola del cultivo de aguacate hass un producto de alta demanda a nivel nacional e internacional. Para el Departamento Administrativo Nacional de Estadística \citet{dane2016cultivo}, Las problemáticas a tener en cuenta en el cultivo del aguacate Hass corresponden a: factores atmosféricos relacionados con la temperatura, las precipitaciones, el viento, la altitud, los factores de las condiciones del terreno y los factores relacionados con la siembra donde se encuentra la fertilización, los abonos y el tratamiento de las plagas y enfermedades.\\

Dentro de las enfermedades más importantes en el cultivo del aguacate Hass se encuentran 
la Stenoma catenifer y el heilipus lauri, insectos y larvas que introducen sus huevos provocando el daño en las semillas de los frutos en crecimiento. Además, el Stenoma catenifer impacta en el fruto al perforar el brote terminal y los laterales del aguacate, formando túneles de hasta 25cm, corta los pedúnculos y la base de los frutos pequeños, como resultado los frutos verdes y pequeños caen.\\

Para prevenir y controlar estas enfermedades, se recomienda implementar prácticas agrícolas adecuadas, como el manejo integrado de plagas, la selección de variedades resistentes y el control de la humedad en el suelo. Además, se deben realizar monitoreos constantes para detectar y tratar a tiempo cualquier enfermedad que pueda aparecer en el cultivo del aguacate Hass.\\

Para la detección de este tipo de plaga en la producción agrícola del cultivo del Hass existe el método Manejo Integrado de Plagas (MIP) en el cual se señala el monitoreo de manera manual y de observación constante partiendo de tres elementos, el primero corresponde a la prevención como cuidados, restricciones y limpieza del personal y sus utensilios de trabajo, el segundo al control donde se utiliza evaluaciones y registros manuales, instalación de trampas y el tercero es el manejo de la enfermedad en el cual se genera una protección y cuidado de las plantas dependiendo de los patógenos dañinos \citet{ica2012manejo}.\\

En este sentido el Machine Learning permite a través de imágenes el reconocimiento de patrones de concentración y expansión de las plagas de manera óptima en todo el cultivo generando una reducción económica y mejorando la calidad del producto agrícola.\\\\

El cultivo del aguacate Hass en Colombia ha tenido una gran demanda a nivel nacional e internacional, generando un crecimiento del 34\% del total de área sembrada de aguacate, además al ser un producto que presenta una cosecha constante por las condiciones del relieve y climáticas del país viene en un crecimiento de área sembrada de un 65\% (Ministerio de Agricultura y Desarrollo Rural, 2021). Por esta razón, la metodología MLOps puede mejorar la calidad, la confiabilidad y la eficiencia de los modelos de machine learning a encontrar el tipo de plaga, la cantidad de daño, identificando procesos de deformación y tipos de coloraciones especificas del área afectada, debido a que los modelos están sujetos a rigurosos procesos de control de calidad y proporciona trazabilidad y transparencia en todo el ciclo de vida del modelo.\\

La utilización de MLOps se ha convertido en una práctica cada vez más extendida en el campo de la ciencia de datos, y se ha demostrado que mejora significativamente la eficiencia y efectividad en la implementación de modelos de Machine Learning \citet{geron2019hands}. Su aplicación en el contexto de la agricultura y la predicción de enfermedades puede ser un paso importante para mejorar la productividad y sostenibilidad del cultivo de aguacate Hass y otros cultivos.\\

Es importante destacar la relevancia de utilizar prácticas de MLOps para garantizar el correcto desarrollo, implementación y mantenimiento continuo de un modelo de control y cuidado de plagas permitiendo mejorar los procesos productivos agrícolas en Colombia. El modelo de Machine Learning al ser un programa de automatización y actualización constante de sus tareas avanza en el mejoramiento y la eficiencia de su procesamiento de información de manera continua a través de la metodología MLOps.
\\\\



% \textbf{TIP:} Contexto + antecendents  + situación problema
% \textit{qué} o \textit{cómo}

\subsection{Formulación del problema}
En este contexto la investigación busca desarrollar una herramienta digital que ayude a pronosticar y prevenir la posible la presencia o no de las enfermedades como el Stenoma catenifer y el heilipus lauri en el cultivo de aguacate Hass, entendiendo que es crítico la detección temprana y continua del brote en un cultivo y por lo tanto la solución debería estar disponible en un software para los científicos de datos, de allí que se planteó la siguiente pregunta de investigación ¿Cómo el uso de MLOps para desarrollar un modelo de Machine Learning permite la integración, la actualización y el despliegue continuo del reconocimiento de las enfermedades Stenoma catenifer y heilipus lauri en el cultivo de aguacate Hass, contribuyendo a mejorar los modelos agrícolas de forma automática y brindando beneficios económicos y sociales a la comunidad de científicos de datos? asimismo ¿Cómo mantener el programa de Machine Learning de manera automatizada y de supervisión continua de manera que no pierda rendimiento?
